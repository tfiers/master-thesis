
\chapter{Introduction}


This thesis discusses the problem of real-time sharp wave-ripple (SWR)
detection.

The first three chapters unpack this sentence. What are sharp wave-ripples?
Why do we care about them, and why would we want to detect them in real-time?
And finally, what are the solved and outstanding engineering problems of
real-time SWR detection?



% We first give an overview of why we did the work described in this thesis. The subsequent paragraphs explain this in a bit more detail; including one paragraph on possible applications. Next, we describe prior work done on this topic. Finally, we lay out how this thesis is structured.


% \section{Why this work}

% \subsection{Overview}

% % Actually also focus on general BCI applicability (at least in abstract)

% The brain is a vast, uncharted territory when it comes to mechanistic understanding. % Neuroscience ventured unusually deep when it discovered 
% The discovery of \emph{place cells} (see below) has been unusually
% satisfactory in this regard. We now know how mammalian brains represent thir
% location in space, where the neurons that do this are located exactly, and
% how the firing rates of these neurons should be interpreted to calculate
% where the animal is in the environment.

% % Cool 'uncharted territory' picture:
% % http://static1.uk.businessinsider.com/image/5979f5039d09184e5a538b4d/legendary-investor-byron-wien-says-the-stock-market-is-entering-uncharted-territory.jpg


% These are neurons whose firing rate encodes the position of  of the anima

% % which neurons

% (see further, \cref{sec:place-cells})


% \subsection{Probing the brain}

% There is a glaring gap in our understanding of the nervous system. We have a
% passable understanding at the circuit level of the first few stages of
% sensory processing, and analogously of the last few stages before muscle and
% gland control. Our understanding of what happens in between those extremes however, is very unsatisfactory.

% The working model of the brain 
% - A few, non connected pathways from the senses. Few recurrence.
% - Higher order processing, joint, recurrence
% - [Output]

% \includegraphics[width=\textwidth]{intro/nervous-system-sketch}

% \begin{figure}
% % \includegraphics[width=\textwidth]{example-image-A}
% \includegraphics[width=\textwidth]{intro/nervous-system-sketch}
% \caption{A naive high level view of the nervous system.
% Some notable features: 1) Although there is a high level of recurrence, not every neuron is connected to every other neuron. 2) The different coloured neurons indicate that not all neurons have the same transfer function and dynamics.}
% \end{figure}

% This, at least, is my understanding of the state of neuroscience after my brief

% Why do I say this? 
% - Purves \& Kandel chapters
%     Organise them according to content (count synapses!! Shortest path!!):
%     - Input \& Output
%     - Higher level (psychiatry, MRI)
%     - Non-specific (synapses, cells, genes)
% - Course 'neural computation' and 'neural systems \& circuits
% - Free reading, and seminars at NERF.

% % Replay is one of the few 

% \acrfull{swr}

% \begin{figure}
%     \begin{subfigure}{0.6\textwidth}
%         \caption{Information flow through a closed-loop neurostimulation system}
%         \label{fig:Closed_loop_system}
%         % \includegraphics[width=\textwidth]{Closed_loop_system}
%         \includegraphics[width=\textwidth]{example-image-A}
%     \end{subfigure}
%     \begin{subfigure}{0.6\textwidth}
%     	\caption{Signals at different stages of the closed loop, and the latencies between them.}
%         \includegraphics[width=\textwidth]{example-image-B}
%     \end{subfigure}
%     \caption
%     { \subref{fig:Closed_loop_system} Simplified data path through a general closed-loop neurostimulation system. The ``stimulating electrode'' may also be an optical fiber, when optogenetic rather than electrical disruption of neural activity is performed. ALU: Arithmetic Logic Unit.}
%     \label{fig:latency}
% \end{figure}

% % ALU: A combinatorial digital electronic circuit that performs operations like addition and comparison on its inputs. Strictly speaking only for integer inputs, but here taken to also operate on floating point numbers.


% \subsection{Hippocampal systems neuroscience}

% \emph{Systems neuroscience} is the study of biological brains at the level of circuits formed by multiple neurons. This is a coarser level than the individual proteins and neurotransmitters studied by molecular biologists, and a more detailed level than fMRI\footnotemark[1] or EEG\footnotemark[2] studies, where each voxel or each electrode represents the entangled activity of many thousands of neurons \cite{Frahm1993,Kreczmanski2007,Collins2016}. Due do its systems approach, it is arguably the most `engineering-like' of all neuroscience-related disciplines. (Though see the papers ``Could a neuroscientist understand a microprocessor?''\cite{Jonas2017} and ``Can a biologist fix a radio?''\cite{Lazebnik2002} for a detailed reflection on the differences in approach between `classical' engineering disciplines, and systems neuroscience or systems biology).

% \footnotetext[1]{Functional magnetic resonance imaging.} %todo
% \footnotetext[2]{Electroencephalography. What is left of the vector sum of many thousands of neurons' electric field potentials, after they have been attenuated by the skull and scalp.}

% Within systems neuroscience (but also in other disciplines of neuroscience), the \emph{hippocampus} is perhaps the most intensively studied brain region. The reasons for this are historically

% its role in learning and memory \cite{HCB_function}, and its critical
% involvement in common diseases such as Alzheimer's and epilepsy \cite{HCB_disease}.

% Most 

% Picture a rodent in a lab environment, placed in an open `arena' or in a simple "maze" (like in \cref{fig:lab-env}). This is most often the (sometimes implicitly) assumed setting in the field of systems-neuroscience hippocampus research. In the following sections, we will assume this setting.

% % Also: Kandel (Nobel and _the_ book) mainly worked on memory


% \subsection{Place cells}
% \label{sec:place-cells}

% \emph{Place cells} are neurons that fire at an increased rate when the ani a particular location in the environment.

% \begin{figure}
%     \begin{subfigure}{0.5\textwidth}
%         \includegraphics[width=\textwidth]{example-image-A}
%     \end{subfigure}
%     \begin{subfigure}{0.5\textwidth}
%         \includegraphics[width=\textwidth]{example-image-B}
%     \end{subfigure}
%     \caption
%     { Rodents in experimental environments [photographs] }
%     \label{fig:lab-env}
% \end{figure}

% \subsection{Replay}

% \subsection{Sharp-wave ripples}

% \subsection{Closed-loop SWR disruption}

% \subsection{Relevance of the experimental setup}
% % Generalisability

% % Replay of other encoded data -- we just haven't decoded anything else besides place.

% \subsection{Other applications}

% \section{Prior work}

% \section{Structure of this thesis}

% A feedforward artificial neural network is a useful first order approximation for the sensory processing pipeline in biological brains. Each sensory experience corresponds to a distinct activation pattern of the network. A life episode of say a few seconds long is a concatenation of experiences, and corresponds to a sequence of such neural activation patterns (which we will call a sequence of `states'). Furthermore, as the raw sensory information advances further through the network, its representation becomes more compressed and abstract.

% One of the areas at the end of this abstraction pipeline in human and other mammalian brains is the hippocampus. In-vivo multi-electrode recordings inside the brain allow us to determine the activation time series for sets of individual neurons. This type of neuroscientific research, mostly in freely behaving mice and rats, 

% % Neuroscience labs around the world study the brain in health and in disease.
% % An anatomical region of particular interest is the \emph{hippocampus}, for
% % its role in learning and memory \cite{HCB_function}, and its critical
% % involvement in common diseases such as Alzheimer's and epilepsy
% % \cite{HCB_disease}.

% \Cref{fig:experiment} outlines one type of experiment used to study the
% hippocampus \cite{ego2010disruption,jadhav2012awake}.

% \begin{figure}
% \centering
% \includegraphics[width=\textwidth]{example-image-a}
% \caption
% { The type of neuroscientific experiment that is made more rigorous by the
%   work described in this paper.
%   % Learning is disrupted in a model animal by injecting a
%   % current pulse in the hippocampus on detection of a sharp-wave ripple event.
% }
% \label{fig:experiment}
% \end{figure}
