
\section{Accuracy}


A sharp wave-ripple detection algorithm needs to make accurate detections. That is, it needs to detect all sharp wave-ripples present in an LFP signal, but nothing more.

To be able to quantify this, we need a baseline labelling that specifies where `true' SWR events are present in the LFP signal. \Cref{sec:reference_labelling} describes how this can be obtained.

% ref next two subsections too

% The accuracy of a detection algorithm on a certain LFP signal



\subsection{Creating a reference labelling}
\label{sec:reference_labelling}

SWR's are not clearly defined. They are an empirical pattern, embedded in a signal that is a complex mix of different sources. Most publications that describe sharp wave-ripples simply give a short qualitative description, and show a trace of a clear SWR complex \cite{Buzsaki2015}. When SWR's are quantified, this is invariably done using the band-pass based method described in \cref{sec:offline}. This means that ``sharp wave-ripple'' events are equated with 

In other words, this method only considers the ripple of a sharp wave-ripple event, and not the sharp wave. This equates 

We can however imagine that an LFP signal may contains SWR events 


% This labelling can be obtained as follows. 

% First, mark the start and end times of ripple-frequency segments using the method described in \cref{sec:offline}.

We now have a set of timestamps that signify where, according to experts,  SWR events are present in the LFP signal. We have one timestamp per SWR event, more or less arbitrarily located within the event. The accuracy quantification as described below requires not one timestamp per event, but two: both a start and an end time. These can easily be obtained using the method described in \cref{sec:offline}, applied to short fragments of LFP signal centered on the expert timestamps.

For 

\begin{figure}
\img[1]{labelface}
\captionn{User interface for SWR labellers}{Each event in the list at the top is represented by two voltage traces: one from an electrode in the pyramidal cell layer (the top trace), and one from an electrode in the stratum radiatum.
The large plot at the bottom gives more comprehensive view of the current event: all 16 channels are plotted (instead of only two), and the plot ranges from 1000 ms before to 1000 ms after the event.
In this large plot, the blue vertical line marks the currently active event. The grey vertical lines correspond to other detected events. The scalebar at the beginning of this plot represents 1 mV.}
\end{figure}


\subsection{Stimulation effect duration}





\subsection{Quantifying accuracy}

We say that an SWR event is succesfully detected when there is an overlap 

% We could also say that the start of the event was detected.
% i.e. t_baseline
% 
% Maybe go for 



% The two relevant measures to quantify the performance of a detect
% 
% Sensitivity = TP / P
% 			  = how many ground truth events are detected
% Precision = TP / (TP + FP)
% 			= how many detections are correct
