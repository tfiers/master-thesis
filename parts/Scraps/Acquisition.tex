
\chapterr{Signal Acquisition}{How we measure}


% amplify

% digitise

% \section{Undiscussed issues}

% Foreign body reaction to electrodes
% Stimulation regime
% Electrode design
% Experimental design



% Earlier drafts:

% \section{Anatomy}
% Schaffer collaterals nice image in:
% p. 399 in Celullar & Molecular Neurophysiology
% and Altman J, Brunner RL, Bayer SA (1973)
%
% \begin{figure}
% \img[0.8]{Human_HC}
% \captionn{The hippocampus in the human brain}{Coronal brain slice from a 20-year old male, stained with cresyl violet to colour cell bodies. The cut is made approximately at the level of the anterior part of the ears ($\approx$ \SI{3}{\cm} posterior to the anterior commisure). [...] The atlas \cite{Mai2015} was used as a guide while marking the locations of different brain areas.}
% \end{figure}
%
% Human HC:
% http://zoomablebrain.bio.uci.edu
% std3 coronal slice 2391
% 
% Human cells:
% http://www.thehumanbrain.info/brain/sections/microslices/A58-r3-0205/index.html
% +31.8 mm

% \section{Measurement}
% The metal is initially uncharged, i.e. it contains equal amounts of positivelattice cores and free floating electrons. Assume that possible externalelectric fields acting on the wire cancel out. The metal is then unpolarised,i.e. the density of electrons is the same everywhere. There are no netelectric fields, and the potential is therefore zero everywhere in the metal.
% The wire is then introduced into the brain. The exposed metal at the wire end(the `electrode') comes in contact with the extracellular fluid. Suppose thatthere is a positive charge surplus in the fluid around the electrode (i.e. apositive voltage). The metal then becomes polarised, with electrons shiftingtowards the electrode end. This results in a positive charge surplus at thereadout end of the wire, outside the brain.
% Electrons are then displaced in the metal, away from the electrode, into the wire. The electron surplus deeper in the wire yields  An attosecond-scale feedback mechanism of new electric fields deeper in the wire (pointing towards the electrode) and electron displacement
% In almost all measuring setups, the readout end of the wire is connected tothe gate of a MOSFET\footnotemark{} (specifically, one that is part of a`buffer' circuit). The charge surplus at the outside end of the wire istherefore placed all the end at the gate contact of the MOSFET. There, thispositive charge emanates an electric field, modulating the conductivity ofthe other two contacts of the MOSFET.
% \footnote{Metal-oxide-semiconductor field-effect transistor. A three-contactelectronic device that lies at the heart of the digital world. The electricfield generated by the charge on the insulated `gate' contact modulates theconductivity between the other two contacts.}
% An electric field is then felt at the electrode, pointing out of the wire. This field drives a chemical reaction at the fluid-metal interface\footnotemark{} that introduces extra electrons into the metal. This continues until the potential difference between the fluid and the electrode is neutralised.