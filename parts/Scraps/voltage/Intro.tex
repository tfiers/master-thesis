Sharp wave-ripples are patterns observed in voltage recordings of the mammalian brain. This chapter describes how such recordings are made.

Most literature on electric potentials in the brain describes them as being the result of currents \cite{Nunez2006, Buzsaki2012a, Leung2011, Linden2014}. More accurate would be to say that they are the result of charge distributions; and that currents (i.e. changes over time of charge locations) give rise to changes over time of the electric potential.

In the first two sections of this chapter, we will show that the electric potential in the brain can be described as follows. Each free charge (like a \ce{Na+} or \ce{Cl-} ion) emits a potential field that decays as $1/r$ (where $r$ is the distance from the charge). This potential field is positive around positive ions, and negative around negative ions. All these potential fields summate linearly, to form the total electric potential $\phi(\loc,t)$ at each location $\loc$ and time $t$.

% This formulation might be slightly controversial. To quote Nunez from the book \emph{Electric fields in the brain}: ``In biologic tissue, current sources at membrane surfaces rather than charges are the so­ called generators of EEG.''

% The popular idea that ``currents cause potentials in the brain'' can be attributed to a macroscopic modelling `trick', which we will shortly describe. We then 

Finally, we give an overview of the technology used to record electric potentials in the brain.



% Nog Nunez:
%
% Knowing the usual elementary relation between charges and electric fields is, by itself, of minimal use in electrophysiology. Static membrane charge produces no electric field in tissue at the macroscopic distances of interest in EEG. The reason is that many other charges in the conductive medium (Na, CI, and other ions) change position so as to shield membrane charge, thereby making the electric field essentially zero at macroscopic distances from any local charge accumulation. This charge shielding issue has often been misunderstood by cognitive scientists and electro physiologists (and even some engineers and physicists), apparently because their physics education emphasized charges as electric field sources. 