\section{Introduction}
\label{sec:GEVD_overview}

In this chapter, we search for a linear combination of channels that yields
an output signal $y_t$ useful for sharp wave-ripple (SWR) detection. More
precisely, we search for a vector $\w \in \reals^C$ in channel (or electrode)
space to project the samples $\z_t \in \reals^C$ on, so that the signal
%
\begin{equation}
\label{eq:linear}
y_t = \w^T \z_t
\end{equation}
%
has high variance (or power) during SWR events, and low variance outside
them. This principle is illustrated with a two-dimensional toy dataset in
\cref{fig:GEVD_principle}. (We can then detect SWR events using threshold
crossings of the envelope of $y_t$, as discussed in \cref{sec:SOTA}).


\begin{figure}
\includegraphics[width=0.6\textwidth]{GEVD_principle_scatter}
\includegraphics[width=0.6\textwidth]{GEVD_principle_strips}
\captionn{Signal-to-noise maximisation via the GEVD}{Toy example to
illustrate the generalised eigenvalue decomposition (GEVD) principle for
signal detection. \emph{Left}: multi-channel time-series data plotted in
`phase space' (meaning without time axis), with blue dots representing
samples where the signal was present, and orange dots representing samples
where it was not. Actually toy data drawn from two 2-dimensional Gaussian
distributions with different covariance matrices. Red vector: first
eigenvector of the signal covariance matrix (also known as the first
principal component). Green vector: first generalised eigenvector of the
signal and noise covariance matrices. \emph{Right}: Projection of both data
sets on both the ordinary eigenvector (``PCA'') and the generalised
eigenvector (``GEVD''). The ratio of the projected signal data variance
versus the projected noise data variance is maximised for the GEVD case.}
\label{fig:GEVD_principle}
\end{figure}


We assume the input signal $\z_t$ to be zero mean. This is achieved with a
straightforward preprocessing step in offline SWR detection, and can be
approximated in online SWR detection by keeping track of a running mean of
$\z_t$, such as an exponentially weighted moving average.

This is a supervised (or data-driven) method: we need to have access to
training data $\z\train_t$, and an associated labelling $x\train_t$ which
marks when there is an SWR event present in $\z\train_t$, and when there is
not. These can then be used to find a good weight vector $\w$, that can then
be applied to detect SWR events in unlabelled data $\z\test_t$.

The algorithm as described above is also a purely spatial filtering method
(at each timestep $t$, only current information from the different channels
is used in calculating the output $y_t$, without incorporating temporal
information from previous timesteps $t_p < t$). The algorithm can be adapted
to also incorporate temporal information, by defining a vector
$\z^\text{stack}_t \in
\reals^{C \cdot P}$, which consists of stacked sample vectors (each
consisting of $C$ channels) from $P$ different timesteps $t_p \leq t$. The
linear weights $\w^\text{stack} \in \reals^{C \cdot P}$ are then calculated
in the same way as for $\w$.

The following sections describes how the vector $\w$ can be found.
