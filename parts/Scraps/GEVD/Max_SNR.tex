\section{Signal-to-noise maximisation}
\label{sec:GEVD_SNR}

Suppose all training samples $\z\train_t$ are gathered and divided over two
data matrices $\Sm \in \reals^{C \cross N_s}$ and $\Nm \in \reals^{C \cross
N_n}$, where $\Sm$ (for `signal') contains all $N_s$ samples of $\z\train_t$
where an SWR was present, and $\Nm$ (for `noise') contains all $N_n$ other
samples. (These matrices can be constructed by simply concatenating all
segments in $\z\train_t$ where an SWR was present, and all segments where
there was no SWR present).

As described in the previous section, we project the data onto the weight
vector $\w$. For the data matrices, this becomes, in vector notation:
%
\begin{align*}
\y_s &= \w^T \Sm \\
\y_n &= \w^T \Nm,
\end{align*}
where each element of the row vectors $\y_s$ and $\y_n$ corresponds to one
(multichannel) sample of $\Sm$ or $\Nm$. \Cref{fig:GEVD_principle} (right)
shows the distribution of the values in two example data vectors $\y_s$ and
$\y_n$.

We want to find the weight vector $\what$ that maximises the variance of
$\y_s$ versus the variance of $\y_n$, i.e.
%
\begin{align}
\what &= \argmax_{\w} \frac{\Var(\y_s)}
                           {\Var(\y_n)}  \nonumber \\[1em]
      &= \argmax_{\w} 
         \frac{\frac{1}{N_s} \y_s \y_s^T}
              {\frac{1}{N_n} \y_n \y_n^T}   \label{eq:argmax_SNR} \\[1em]
      &= \argmax_{\w}
         \frac{\frac{1}{N_s} \w^T \Sm \Sm^T \w}
              {\frac{1}{N_n} \w^T \Nm \Nm^T \w}  \nonumber
\end{align}

The signal and noise covariance matrices $\Rss \in \reals^{C \cross C}$ and
$\Rnn \in \reals^{C \cross C}$ are defined as:\footnotemark{}
%
\begin{align}
\Rss &= \frac{1}{N_s} \Sm \Sm^T \label{eq:covariance_s}\\
\Rnn &= \frac{1}{N_n} \Nm \Nm^T \label{eq:covariance_n}
\end{align}
%
($\Rss$ and $\Rnn$ are symmetric. The diagonal elements yield the variance of
each channel, and the off-diagonal elements yield the covariance between
pairs of channels).

\footnotetext{To be precise, these are only estimates of the `true'
covariance matrices of two random, \emph{ergodic} variables $\z_{s,t}$ and
$\z_{n,t}$, of which the data in $\Sm$ and $\Nm$ are some samples.
\Cref{eq:covariance_s,eq:covariance_n} are then correct definitions only in
the limit for $N_s \to \infty$ and $N_n \to
\infty$.}

The expression for the optimal weight vector (\cref{eq:argmax_SNR}) then
becomes:
%
\begin{equation}
\label{eq:argmax_R}
\what = \argmax_{\w}
       \frac{\w^T \Rss \w}
            {\w^T \Rnn \w}
\end{equation}
%
This quotient of quadratic forms is the `generalised Rayleigh quotient' of
$\Rss$ and $\Rnn$. It reaches its maximum when $\w$ is the first so caled
`generalised eigenvector' of $\Rss$ and $\Rnn$. The following section defines
these terms, and proves the ``$\argmax$'' --- ``first generalised
eigenvector'' correspondence.
