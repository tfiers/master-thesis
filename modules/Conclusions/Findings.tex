\section{Summary of findings}

\begin{figure}
\img{PR-and-latency--0_7}
\captionn{Performance comparison of the best SWR detectors}{Legend as in \cref{fig:BPF-performance}. ``Proposed online BPF'' is the best representative of the current generation of SWR detectors (band-pass filters on a pyramidal cell layer channel; see \cref{ch:BPF}). ``GEVec'' and ``Recurrent neural network'' (RNN) are the newly investigated, data-driven algorithms. Using the GEVec algorithm with about ten delay samples (corresponding to 10 ms at $f_s = 1000$ Hz) yielded optimal performance (see \cref{ch:GEVec}). Both a version of the GEVec algorithm using multiple, spatially distributed input channels (``all ch.''), and a version using the same, single input channel as the band-pass filter (``pyr ch.'') are presented here. The RNN here used all input channels. \newline
The GEVec-based detectors and the band-pass filter have very similar performance. The GEVec-based detectors have very slightly lower latency (Difference in medians of relative latency is $\pm$ 6 percentage-points. Difference in medians of absolute latency is 2 milliseconds). In the high recall regime (at $R > 93\%$), the band-pass filter is slightly more precise (less false positive detections). The single-channel and the multichannel GEVec-based detectors show near identical performance.}
\label{fig:PR-and-latency-conclusion}
\end{figure}

We compared different existing online ripple detection filters from the literature. We found important differences between these filters (see \cref{sec:recommend}). We proposed a band-pass filter (labelled ``Proposed online BPF'' in \cref{fig:PR-and-latency-conclusion}) that is at least as good as the filters from the literature, and used this as a baseline for comparison. Additionally, we defined a framework to quantify the performance of online SWR detection algorithms.

We then investigated, for the first time, two data-driven, multichannel algorithms for SWR detection. The first, based on the generalized eigenvector (GEVec) decomposition, achieves performance roughly on par with the best online band-pass filter (BPF); see \cref{fig:PR-and-latency-conclusion}). Additionally, its output envelope is very similar to those of online band-pass filters, suggesting that the GEVec-based algorithm finds filters similar to those of the explicitly designed online band-pass filters.

The GEVec-based algorithm uses multiple past input samples of the LFP to calculate the current output sample. We found that using about ten such delayed samples (corresponding to 10 ms at $f_s = 1000$ Hz) yielded optimal performance.

The GEVec-based algorithm has a very slightly lower detection delay than the proposed online BPF, while the online BPF yields a slightly lower amount of false positives in the high recall regime. These differences might not hold up when analyzing data from multiple recordings, animals, and electrode setups, however.

To investigate the use of multiple input channels, we also applied the GEVec-based algorithm to a single channel of input data. This yielded an online SWR detector with near identical performance and output envelopes as the multichannel GEVec-based detector. This suggests that, at least for this algorithm and under the constraints discussed in \cref{sec:limits}, using multiple, spatially distributed input channels is not advantageous for online SWR detection.

The second algorithm investigated, a type of recurrent neural network (RNN), showed worse detection performance than the band-pass and GEVec-based filters, under the used evaluation scheme (see \cref{fig:PR-and-latency-conclusion}). Analyzing its output envelopes however (see \cref{fig:RNN-envelopes,fig:RNN-profile}) reveals an interesting finding. Although the RNN was only trained on segments with strong ripple-oscillations, its output also yields high-confidence detections whenever the low-frequency spatiotemporal signature of an SWR is present (as described in \cref{ch:SWR} on SWR anatomy and physiology), but without necessarily any strong ripple oscillation present.

% - We can only marginally improve sota
%   (2 ms d median; ~6% percentage points; while high-recall PR performance is slightly worse).
% - Real (first?) quantification and comparison of band-pass filters                               
% - 10 delays / ms best for gevec
% - Take care with BPF: possible to introduce unnecessary (group) delay.
% - Using more than one electrode for detection (such as a stratum radiatum electrode in addition to the stratum pyramidale electrode, or multiple stratum pyramidale electrodes) does not significantly improve accuracy nor latency.
