\section{Result for SWR detection}

We divided the 34-minute long LFP recording into two datasets. The first 60\% was used as training data, to calculate the covariance matrices $\Rss$ and $\Rnn$, and to calculate from these the optimal linear combination of channels $\what$, as described in the preceding sections. The remaining 40\% was used to evaluate this filter $\what$, and to compare it to the state-of-the-art method (the single-channel online band-pass filter).\footnotemark{}

\footnotetext{The 60-40 division was chosen arbitrarily -- under the constraint that there are sufficient amounts of both training and test data.}

\begin{figure}
\img[1.1]{LSM-comp}
\captionn{Linear, SNR-maximising combinations of electrodes}{
  \panel{A} Example input and output signals.
    \Top: multi-channel LFP, $\z_t$. Light-blue vertical band: a reference SWR segment.
    \Bottom: output envelopes $n_t$, for different filtering algorithms. Dashed horizontal lines: detection thresholds, chosen so that each algorithm reaches a recall value of 80\%. Green triangles: correct detections. Red triangles: incorrect detections. Brackets indicate envelope range (min, max) over the entire test set.
  \panel{B} Generalized eigenvector $\what$ (i.e. the weights of the multichannel filter), for a purely spatial filter.
  \panel{C} Generalized eigenvector $\what$ for a spatiotemporal filter with a one-sample delay.}
\label{fig:LSM-comp}
\end{figure}

\Panelref{fig:LSM-comp}{A} shows an excerpt of the test input signal, and the corresponding filter output envelopes (blue for state-of-the art method, orange for GEVec-based multichannel method. Filter outputs $o_t$ are rectified to obtain envelopes $n_t = \abs{o_t}$). Additional excerpts are shown in \cref{fig:GEVec-extracts}. The elements of $\what$ (i.e. the filter weights) are visualized in \panelref{fig:LSM-comp}{B}.

It is clear that the GEVec filter output indeed has high power during SWR events, as promised by the theoretical derivation. The filter weights and signal excerpts reveal that the GEVec output is composed mainly of a few channels in the stratum radiatum (channels 4-5-6 here), where they pick up the sharp waves. However, the filter output envelope is also high for sharp wave-like activity on these channels, without or with only very weak ripple activity in the higher channels: see \cref{fig:GEVec-extract-a,fig:GEVec-extract-b,fig:GEVec-extract-d}. This results in false positive detections.

We also notice some premature detections (\cref{fig:GEVec-extract-a,fig:GEVec-extract-c}), where the sharp wave -- and thus also the GEVec filter output -- already has high power before the corresponding ripple has started. The GEVec detection then happens before the start of the reference SWR-segment, which is based on ripple power. These early detections thus count (arguably unfairly so) as false positives, under the evaluation scheme that we use.

\begin{figure}
\img[0.94]{PR-and-latency}
\captionn[]{Sensitivity, precision \& latency tradeoffs, }{for different linear filters \& for a range of thresholds.\newline
Each threshold setting for an algorithm corresponds to a point on the precision-recall curve in the bottom-left panel, and to a distribution of relative detection latencies. The median and interquartile range of this distribution are plotted in the top-left or bottom-right panel (as a point of the bold curve, and a slice of the shaded band, respectively).\newline
These latency distribution plots are divided over two panels so that their entire range can be clearly visualized, both for the low recall -- high precision regime as for the high recall -- low precision regime. The cutoff is made at the point where recall equals precision (AKA the $\max{F_1}$ point), marked with shaded black circles.}
\label{fig:LSM-PR-and-latency}
\end{figure}

This effect, where the sharp wave is discernible before the ripple, results in faster detections when the detection \emph{does} fall within the reference segment. When all 468 SWR events from the test set are analysed, we find a large improvement in detection latency: at thresholds where both methods detect 80\% of these reference SWR segments, the median absolute detection latency drops from 24 ms for the state-of-the-art online band-pass filter to 12 ms for the GEVec-based filter. The relative detection latency drops by 32.5 percentage points, from 58.1\% to 25.6\%. Similar latency improvements are found for other recall and precision values: see \cref{fig:LSM-PR-and-latency}.

This strong improvement in detection latency trades off with an increase in false positives, as was already observed qualitatively. At the aforementioned sensitivity of 80\%, the state-of-the-art method has a precision of 94\%, whereas the GEVec-based method has a precision of only 63\% (i.e. more than a third of detected events are classified as false positives). This strong decrease in precision is true over the entire $PR$-curve: see \cref{fig:LSM-PR-and-latency}.
