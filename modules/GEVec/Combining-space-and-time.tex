\sectionp{1p}{Combining space and time}
\label{sec:spatiotemporal}

It is hardly surprising that the GEVec-based algorithm as described above cannot discern ripple activity (which is by definition a temporal pattern), as the algorithm is a purely \emph{spatial} filter: at each timestep $t$, only current information from the different channels is used in calculating the output $o_t$, without incorporating temporal information from previous timesteps $t_p < t$.

The GEVec method can be easily adapted to also incorporate temporal information however, by defining a vector $\z\stack_t \in \reals^{CP}$ which consists of stacked sample vectors (each consisting of $C$ channels) from $P$ different timesteps $t_p \leq t$. The linear weights $\w\stack \in \reals^{CP}$ used to obtain the output signal $o_t = (\w\stack)^T \z\stack_t$ are then calculated analogously to the purely spatial filter, i.e. as the first generalized eigenvector of the ordered pair $(\Rss\stack, \Rnn\stack)$, with both covariance matrices $\in \reals^{CP \cross CP}$.

Adding just one such delayed time step (i.e. $P = 2$) yields a major performance improvement (see \cref{fig:LSM-PR-and-latency}): the precision-recall curve now approaches (but not does not reach) the PR-curve of the state-of-the-art algorithm, while the latency improvements of the `no delay' GEVec algorithm are mostly retained. precision-recall-latency tradeoff and algorithm comparison is shown in \cref{fig:LSM-PR-and-latency}. \Panelref{fig:LSM-comp}{C} shows the GEVec $\w\stack$. \Panelref{fig:LSM-comp}{A} and \cref{fig:GEVec-extracts} show example output envelopes (green traces).

These results, particularly the visualized weights in \panelref{fig:LSM-comp}{C}, indicate that this one-delay GEVec filter utilizes spatiotemporal information about both the sharp wave and the ripple.
