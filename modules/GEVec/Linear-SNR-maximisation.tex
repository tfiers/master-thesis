\sectionp{2p}{Linear signal-to-noise maximisation}
\label{sec:LSM}

In this chapter, we search for a linear combination of channels that yields an output signal $o_t$ useful for sharp wave-ripple detection. More precisely, we search for a vector $\w \in \reals^C$ in channel (or electrode) space to project the samples $\z_t \in \reals^C$ on, so that the output signal
%
\begin{equation}
\label{eq:linear}
o_t = \w^T \z_t
\end{equation}
%
has high variance (or power) during SWR events, and low variance outside them.\footnotemark{} This principle is illustrated with a two-dimensional toy dataset in \cref{fig:GEVec_principle}. We can then detect SWR events using threshold crossings of the envelope of $o_t$, as discussed in \cref{ch:eval}.

\footnotetext{We assume that the input signals are zero-mean, such that the power $P$ of the output signal equals its variance:
%
$P_o = \ev{o_t^2} = \ev{(o_t - \mu_o)^2} = \Var(o_t)$ when $\mu_o = 0$, which is the case for zero-mean input channels:
%
$\mu_o = \ev{o_t} = \ev{\w^T \z_t} = \sum_i w_i \ev{z_{t,i}} = 0$ when $\ev{z_{t,i}} = 0$ for all channels $i$.
%
\\This zero-mean assumption is reasonably well fulfilled for the analysed recording: the sample values of a 10-second moving average of the recording are near zero (median -0.02 \uV{}, IQR 0.13 \uV{}. In comparison, the RMS-value of the recording is 210 \uV{}).
\\Input data that is not zero-mean can be readily transformed to be so (even in an online setting), by subtracting a (moving) average from the input signal.}


\begin{figure}
\img[0.4]{scatter}\hfill
\img[0.55]{strips}
\captionn{Linear signal-to-noise maximisation}{Toy example to illustrate the generalized eigenvector approach to signal detection.
\Left: multi-channel time-series data plotted in `phase space' (meaning without time axis), with blue dots representing samples where the signal was present, and orange dots representing samples where it was not. Actually toy data drawn from two 2-dimensional Gaussian distributions with different covariance matrices. Red vector: first eigenvector of the signal covariance matrix (also known as the first principal component). Green vector: first generalized eigenvector of the signal and noise covariance matrices.
\Right: Projection of both data sets on both the ordinary eigenvector (``PCA'') and the generalized eigenvector (``GEVec''). The ratio of the projected signal data variance versus the projected noise data variance is maximised for the GEVec case.}
\label{fig:GEVec_principle}
\end{figure}

The next two sections describe how this vector $\w$ can be found.
