\chapter{Supplemental figures}

\begin{figure}
\img[1.2]{human-cells-comp}
\captionn{Individual neurons in the human hippocampus}{Zoom-in of the top-right panel of \cref{fig:brain-anatomy}. Black box in the top right panel marks location of the bottom panel. Small white box in bottom panel hints at the size of \cref{fig:neuropil}.}
\label{fig:human-cells-comp}
\end{figure}

\begin{figure}
\img[1]{implant-photos}
\captionn{Probe shank \& micro-drive array}{Reproduced from \cite{Michon2016}. \Left: flexible, 40 mm long probe shank. \Right: test subject implanted with so called `micro-drive array'. This is an electrode positioning device surgically attached to the rat's skull. The rat is anesthesised during surgery and is administered pain-killing drugs during the recovery period. The micro-drive array guides multiple electrodes into the brain (such as ``tetrodes'', bendable probes, and reference or stimulation electrodes), and connects them to the recording and stimulation hardware. The depth of the inserted electrodes can be precisely adjusted. This device allows to make voltage recordings while the animal can move around (the so called ``freely behaving'' setup, which allows for more natural behaviour than immobilised recording setups).}
\label{fig:implant-photos}
\end{figure}

\begin{figure}
\img[1]{neuropil}
\captionn{The environment in which recordings are made}{\emph{Figure
adapted from \cite{Knott2008}}. Scanning electron micrograph of a slice of mouse cortex. (Rat hippocampal tissue looks similar, see e.g. \cite{Martin2017}). This type of tissue is called ``neuropil''. It consists of the long and narrow excrescences of neurons (dendrites and axons) and of glial cells, seen here in cross and through section. Note that cell bodies are much larger than the cross sections of dendrites and axons as seen here; a typical neural cell body (``soma'') of 15 \um{} wide would be larger than this image, which is 7 \um{} wide. (See also the small white box in \cref{fig:human-cells-comp}, which is about the same size as this electron micrograph). Some landmarks are the many mitochondria (two of which are annotated with red *); two myelinated axons (green *); and all the lipid bilayers separating cells from their environment (thin black lines, two of which are highlighted in blue). The pink inset (1 \um{} by 1.24 \um{}) shows two ``boutons'' (axon terminals) synapsing onto a dendrite. Note the synaptic vesicles (the many small dark circles), which encapsulate neurotransmitter molecules, ready for release in the synaptic clefts (black and white arrows).}
\label{fig:neuropil}
\end{figure}


\begin{figure}
\subimg{102}{fig:GEVec-extract-a}
\subimg{118}{fig:GEVec-extract-b}
\end{figure}
\begin{figure}\ContinuedFloat
\subimg{120}{fig:GEVec-extract-c}
\subimg{537}{fig:GEVec-extract-d}
\captionn{Extracts from input data and corresponding linear filter output envelopes}{See \cref{fig:LSM-comp} for legend.}
\label{fig:GEVec-extracts}
\end{figure}


\begin{figure}
\img[0.5]{fullrect/198_6--199_2}
\img[0.5]{fullrect/541_2--541_8}\\[1em]
\img[0.5]{fullrect/542_4--543_0}
\img[0.5]{fullrect/606_6--607_2}
\captionn[, ]{False positive RNN detections}{caused by SWR-like spatiotemporal LFP profiles without well-developed ripples.}
\label{fig:RNN-profile}
\end{figure}


\begin{figure}
\img[0.62]{PR/laterstart}
\img[0.62]{PR/supalate}
\captionn{Start-block RNN's}{Online SWR detection performance of GRU-RNN's using only a pyramidal cell layer LFP channel as input. \Left: two-layer RNN. \Right: one-layer RNN. See \cref{fig:BPF-performance} for legend.}
\label{fig:RNN-PR-startblocks}
\end{figure}

\begin{figure}
\img[0.62]{PR/pyr}
\img[0.62]{PR/shallowpyr}
\captionn{Single-input-channel RNN's}{Online SWR detection performance of GRU-RNN's using only a pyramidal cell layer LFP channel as input. \Left: two-layer RNN. \Right: one-layer RNN. See \cref{fig:BPF-performance} for legend.}
\label{fig:RNN-PR-pyr}
\end{figure}
