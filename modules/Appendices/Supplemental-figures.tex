\chapter{Supplemental figures}

\begin{figure}
\img[1.2]{human-cells-comp}
\captionn{Individual neurons in the human hippocampus}{Zoom-in of the top-right panel of \cref{fig:brain-anatomy}. Black box in the top right panel marks location of the bottom panel.}
\label{fig:human-cells-comp}
\end{figure}

\begin{figure}
\img[1]{implant-photos}
\captionn{Probe shank \& micro-drive array}{Reproduced from \cite{Michon2016}. \Left: flexible, 40 mm long probe shank. \Right: test subject implanted with so called `micro-drive array'. This is an electrode positioning device surgically attached to the rat's skull. The rat is anesthesised during surgery and is administered pain-killing drugs during the recovery period. The micro-drive array guides multiple electrodes into the brain (such as ``tetrodes'', bendable probes, and reference or stimulation electrodes), and connects them to the recording and stimulation hardware. The depth of the inserted electrodes can be precisely adjusted. This device allows to make voltage recordings while the animal can move around (the so called ``freely behaving'' setup, which allows for more natural behaviour than immobilised recording setups).}
\label{fig:implant-photos}
\end{figure}


\begin{figure}
\subimg{102}{fig:GEVec-extract-a}
\subimg{118}{fig:GEVec-extract-b}
\end{figure}
\begin{figure}\ContinuedFloat
\subimg{120}{fig:GEVec-extract-c}
\subimg{537}{fig:GEVec-extract-d}
\captionn{Extracts from input data and corresponding linear filter output envelopes}{See \cref{fig:LSM-comp} for legend.}
\label{fig:GEVec-extracts}
\end{figure}
