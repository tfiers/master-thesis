\section{Results for SWR detection}
\label{sec:RNN-results}

The first GRU network that we analyzed used all 16 input LFP channels, had 2 layers of 40 hidden units each, and was trained on a target function $y_t$ that is $1$ during the entire duration of each reference SWR segment, and $0$ elsewhere. \Cref{fig:validloss} shows the evolution of the validation loss while training this RNN. \Cref{fig:RNN-envelopes} shows example output envelopes $n_t$.

\begin{figure}
\img[0.6]{575_4--576_0}
\img[0.6]{589_8--590_4}
\captionn{RNN output envelopes}{See \cref{fig:BPF-outputs} for legend.}
\label{fig:RNN-envelopes}
\end{figure}

We notice that the RNN output matches the target block-shaped function $y_t$ relatively well. Together with the validation loss curve, it seems that the target function was relatively well learnable.

Secondly, the RNN clearly uses more information than the ripple oscillation only to generate its output. In the right panel of \cref{fig:RNN-envelopes}, the RNN outputs a positive with high confidence. The low frequency LFP over all channels indeed has the profile of an SWR; but the ripple is almost non-existent, as can be seen in the band-pass and GEVec filter outputs. This detection of the RNN is thus classified as a false positive. Additional examples of this phenomenon are shown in \cref{fig:RNN-profile}.

% The fact that the trained RNN seemingly uses the low frequency spatiotemporal pattern of SWR's leads to a correct and early detection in the left panel of \cref{fig:RNN-envelopes}, but a correct and relatively late detection in the right panel; this might be because the negative sharp wave only starts after the ripple oscillation has already begun.

This phenomenon, where the RNN seemingly uses the low frequency spatiotemporal pattern of SWR's but does not require clear ripples, leads to a low precision for the RNN detector, under our evaluation scheme. \Cref{fig:RNN-PR} compares the precision-recall curve of this RNN detector with those of previous online SWR detectors, as well as comparing their detection latencies.

\begin{figure}
\img[0.9]{PR_full_fullrect}\\[2.5em]
\img[0.9]{PR_fullrect}
\captionn{Online SWR detection performance of an RNN}{See \cref{fig:BPF-performance} for legend. \Top: Full PR-curve. \Bottom: zoom-in on high-recall/high-precision region.}
\label{fig:RNN-PR}
\end{figure}

The detection thresholds in \cref{fig:RNN-envelopes} are chosen at the maximal $F_1$ point for each output envelope. Because of the relatively low precision (high false-positive rate) of the RNN, this threshold is set high for the RNN envelope. This leads often to relatively late detections; see e.g. the rightmost detection in \cref{fig:RNN-envelopes}.

To investigate the influence of input channels, we trained a GRU-RNN that only used a single input channel (namely the pyramidal cell layer channel that is also used as input for the band-pass filters). The resulting RNN behaves very much like the band-pass filter (\cref{fig:RNN-pyr-envelopes}). It does not show the non-ripple false positive detections of the all-channel RNN. Its detection performance approaches, but is not as good as the baseline online band-pass filter (\cref{fig:RNN-PR-pyr}).

A second net was trained on this single channel, this time with only one hidden layer, of 25 hidden units. Its performance was slightly worse than the two-layer, 40 hidden units per layer RNN (\cref{fig:RNN-PR-pyr}).

Finally, we experimented with a different target function; Namely a $y_t$ that is $1$ only around the start of reference SWR segments. This did not result in good performance for all-channel, two-layer, 40 hidden unit RNN's (\cref{fig:RNN-PR-startblocks}).
