\chapter*{Abbreviations}
\label{abbrs}
\addToTOC{abbrs}

\begin{deflist}
\abbr{BPF}{Band-pass filter}{See \cref{ch:BPF}.}
\abbr{CA1}{``Cornu Ammonis'', subregion 1}{Region in the hippocampus where voltages are recorded from (see \cref{fig:human_hc_zoom,fig:brainzoom}).}
\abbr{CA3}{``Cornu Ammonis'', subregion 3}{Region in the hippocampus (see \cref{fig:human_hc_zoom,fig:brainzoom}). CA3 sends many axons (called ``Schafer collaterals'') to CA1.}
\abbr{FIR}{Finite impulse response}{A linear filter whose output is a convolution of the input signal with some kernel.}
\abbr{GEVal}{Generalized eigenvalue}{See \cref{sec:generalized-eigenproblem}.}
\abbr{GEVec}{Generalized eigenvector}{See \cref{sec:generalized-eigenproblem}.}
\abbr{IIR}{Infinite impulse response}{A linear filter whose output at each timestep is a function of both input samples and preceding output samples.}
\abbr{IQR}{Interquartile range}{A measure of the spread of a set of one-dimensional values, that is robust to outliers. Difference between the 75th and the 25th data percentile.}
\abbr{KDE}{Kernel density estimate}{}
\abbr{LFP}{Local field potential}{The extracellular electric potential (see \cpageref{def:LFP}).}
\abbr{RMS}{Root-mean-square}{$\sqrt{\ev{x_t^2}}$ for a signal $x_t$.}
\abbr{RNN}{Recurrent neural network}{See \cref{ch:RNN}.}
\abbr{SNR}{Signal-to-noise ratio}{See \cref{sec:LSM}.}
\abbr{SOTA}{State of the art}{The algorithm currently used for SWR detection, namely an online single channel band-pass filter.}
\abbr{SWR}{Sharp wave-ripple}{The pattern in the LFP that we want to detect in real-time. See \cref{sec:swr}.}
\end{deflist}
