The sharp wave-ripple (SWR) is a well-known motif in voltage recordings of the mammalian brain, strongly linked to learning and memory. The brain has been shown to replay past experiences during SWRs. A powerful method to study these phenomena is to apply feedback stimulation specifically during SWRs. This requires a `closed-loop' approach where SWRs are detected in real-time. The state-of-the-art algorithm recognises SWRs based on the ripple only, and detects them quite late -- often after a third of the SWR event has already passed. This compromises experimental power, particularly in SWR disruption experiments. In this thesis, we investigate whether new signal processing algorithms can improve the state-of-the-art in online SWR detection; especially when multiple, spatially distributed input channels are used. We find the state-of-the-art method to still be the best when only events with strong ripple-oscillations are considered. When the experimenter wants to detect SWR events with weaker ripples however, the newly presented algorithms prove advantageous.
