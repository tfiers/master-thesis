\chapter*{Symbols}
\label{symbols}
\addToTOC{symbols}



\section*{Notation}

\begin{deflist}
\notation{y}{Scalars are denoted in lowercase italic.}
\notation{\z}{Vectors are denoted in lowercase boldface.}
\notation{\A}{Matrices are denoted in uppercase boldface.}
\notation{\had}{Elementwise multiplication. (``Hadamard product'').}
\notation{\sigmoid(\cdot)}{Sigmoid `squashing' function. $\sigmoid(x) = \frac{1}{1 + \exp(-x)} \qc \in (0, 1)$.}
\notation{\tanh(\cdot)}{Hyperbolic tangent. $\tanh(x) = 2\ \sigmoid(x) - 1 \qc \in (-1, 1)$.}
\end{deflist}



\section*{Signals}

\begin{deflist}
\notation{\z_t}{Digitized LFP sample at discrete time step $t$. $\z_t \in \reals^c$, with $c$ the number of channels (i.e. the number of electrodes simultaneously recorded from). Input to an SWR detection algorithm.}
\notation{o_t}{Output signal of an SWR detection algorithm, $\in \reals$.}
\notation{p_t}{Transformation of $o_t$, so that it is constrained to $\reals^+$. Should be high when the corresponding input sample $\z_t$ is part of an SWR segment, and low when it is not. $\ p_t = \abs{o_t}$ for online linear filters; $p_t = \sigmoid(o_t)$ for the RNN's of \cref{ch:RNN}.}
\notation{y_t}{Binary target signal, used when training data-driven SWR detection algorithms. We define $y_t = 1$ when the corresponding input sample $\z_t$ is part of an SWR segment, and $y_t = 0$ when it is not.}
\end{deflist}
