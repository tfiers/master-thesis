
% Sharp-wave ripple (SWR)
% Local electric field potential (LFP)


% \begin{abstract}
% \inlineHeading{Objective}
% A powerful method to study the brain is to specifically disable one of its aspects. Specificity in time is a challenge however. The events under study must be detected in real-time, and well before they end, for the temporary disturbance to have any pertinent effect.
% [..]
% Experimenters need an SWR detection algorithm that 1) is fast, i.e. has low execution time and recognises an SWR event after only a few samples; and 2) detects all SWR events and nothing else (i.e. has both high sensitivity and high precision).
% \inlineHeading{Approach}
% We propose a new, data-driven algorithm to detect SWR's in real-time, and validate it across multiple animals and species.
% [..]
% \inlineHeading{Main results}
% The new algorithm detects SWR events considerably faster than the state of the art method, while being equally sensitive and precise. [..]
% We provide pre-optimised filter coefficients, so that the proposed algorithm can be implemented in existing real-time SWR detection systems with minimal effort.
% \inlineHeading{Significance} [..]

% \end{abstract}


% One such phenomenon is the sharp-wave ripple, a ... which has been strongly linked to replay of memories.
% A powerful method to understand activity in the brain, is to detect -- in real-time -- each occurence of some neural activity pattern, and then temporarilly perturb an associated brain region, e.g. by electrical or optogenetic means.
% A powerful method to understand neural activity, is to detect an event of interest in real-time, and subsequently apply a temporary perturbation to the studied system.
% Closed-loop experiments are a powerful method to study the brain. The applied perturbations are specific in time and conditioned on detection of the neural activity pattern under study.
% This is challenging when the phenomenon to be perturbed is 

% We want to detect neural activity patterns 

% The start of the event is detected

% An important such event is the sharp-wave ripple

% A part of the brain is temporarily perturbed whenever 

% Neural activity patterns of interest are detected in real-time, which triggers a temporary perturbation 

% Each occurence of a neural activity pattern is detected in real-time, which triggers a temporary perturbation of a set of associated neurons.

% This triggers a temporary deactivation of a set of neurons.


% temporarilly disable a region of the brain with a current pulse or 

% detect some pattern of interest in real-time, and then  


% ___________________________________________________________________________
% ___________________________________________________________________________

% . In the ‘closed-loop’ approach to neural research and therapy, each real-time detection of a motif triggers a temporary brain stimulation. One popular closed-loop experiment is the detection and disruption of ‘sharp wave ripples’ (SWRs), a motif strongly linked to learning and memory.



% The value of using more than one electrode (as in a probe e.g.) [is investigated].

% There is significant opportunity for improvement at the algorithmic level.





% Brain recordings often contain clear motifs, or recurring patterns of activity

% It has a significant detection latency, due to both the inherent filter latency 
% One particular motif, the ‘sharp wave ripple’ (SWR), is of great interest in learning and memory research.


% .., during which spatial trajectories are replayed.

% ..  that is strongly linked to memories. 

% In the closed-loop approach to neural research and therapy, the real-time detection of such a pattern is tied to a temporary manipulation.

% The mammalian hippocampus replays (at ) spatial trajectories.

%  During these 100 ms long activity bursts in the hippocampal electric field, spatial trajectories are replayed in mammals.

% In closed-loop settings for neural research and therapy, these recurring patterns are detected in real time and tied to a specific temporary manipulation.

% Memory researchers want to disable the hippocampus specifically during SWR’s.

% The sharp wave - ripple (SWR) is a 100 ms long activity pattern in the hippocampal electric field that is strongly linked to learning and memory. 

