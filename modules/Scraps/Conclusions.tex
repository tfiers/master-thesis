
\chapter{Conclusions \& discussion}

\section{Summary of findings}

% Main result:
% - We can only marginally improve sota
%   (2 ms d median; ~6% percentage points; while high-recall PR performance is slightly worse).

% Minor:
% - Real (first?) quantification and comparison of band-pass filters
% - We cannot improve using multiple signals from spatially distributed electrodes: multi-channel performs as well as classical single-channel

% Minor minor:
% - Take care with BPF: possible to introduce unnecessary (group) delay.

\begin{figure}
\img{PR-and-latency--0_7}
\captionn{Performance comparison of the best detectors}{Legend as in \cref{fig:BPF-performance}.}
\label{fig:PR-and-latency-conclusion}
\end{figure}

% Using more than one electrode for detection (such as a stratum radiatum electrode in addition to the stratum pyramidale electrode, or multiple stratum pyramidale electrodes) does not significantly improve accuracy nor latency.

% Around 10 delays (~10 ms) is best for GEVec (both single and multichannel)




\section{Limitations of this work}

\section{Recommendations for experimenters}

\section{Directions for further work}

\section{Conclusion}
% Words, communicated either visually or orally, are imprecise and low bandwidth.
