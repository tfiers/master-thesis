
\section{Confounder signals}


The two major events in the neocortical LFP during sleep are ``slow oscillations'' and ``sleep spindles''. Slow oscillations have a frequency between 0.3 and 2 Hz and occur during deep, non-REM sleep (which is also called ``slow wave sleep'') \cite{Buzsaki2015,Bazhenov2006}.
% Intrinsic in cortex.
% In hippocampus also?

Sleep spindles are \range{12}{18}{\Hz} oscillations in the neocortex, that last between 400 and 1000 ms (i.e. there are between 4 and 18 cycles per spindle). They occur during superficial and intermediate sleep stages, and are sometimes observed in the hippocampus, which they reach through the entorhinal cortex \cite{Buzsaki2015,Bazhenov2006}.
%
%   - 12 - 18 Hz oscillation
%       - 9 - 12 Hz more frontally
%   - Generated in thalamus (see Buzsaki 2015, p. 1102)
%       - + in thalamic nuclei, - in thalamic reticular nucleus
%       - Synchronised across thalamic nucleu via neocortical feedforward
%       - Thus also called 'thalamocortical spindles'
%   - ``Loose temporal relationship with SWR's''


