\chapter{The local field potential}
\label{ch:LFP}

This thesis discusses software for processing intracranial voltage recordings. This chapter describes the nature of such voltage signals.

At every location in the brain, the electric field vector $\E$ (in units of N/C or V/m) points in the direction that a positive charge would be pushed towards, if it was placed at that location. The electric potential $\phi$ is defined\footnotemark{} such that:
%
\begin{equation}
\label{eq:potential}
\E = -\grad{\phi},
\end{equation}
%
i.e. such that the electric field points from locations of high potential to locations of lower potential. The electric field potential $\phi$ is a scalar field, in units of V. When it is measured extracellularly, neuroscientists refer to it as the \emph{local field potential} (LFP) -- especially when only frequencies below about 500 Hz are considered.

\footnotetext{The electric field potential $\phi$ is only defined when the electric and magnetic fields do not vary too quickly over time, i.e. when $\pdv*{\E}{t} \approx 0$ and $\pdv*{\B}{t} \approx 0$. It then directly follows from Maxwell's equations that both fields become decoupled, and that the electric field becomes irrotational ($\curl{\E} = 0$), so that an electric potential may be defined as in \cref{eq:potential}. This so called \emph{quasi-static} assumption is met in electrophysiological conditions \cite{Nunez2006,Plonsey2007}.}

We now discuss two models used to estimate the LFP $\phi$. Both models calculate $\phi$ based on the mobile charge density $\rho$ at each location in the brain and at each point in time. ($\rho$ is the net positive surplus of mobile charges, in units of C/\si{\metre^3}. Biological tissue contains a high number of such mobile charges. These are mostly ions, with \ce{Na+}, \ce{Cl-}, and \ce{K+} the most abundant ones \cite{Martinsen2015a}).

Both models consist of a single divergence equation (namely \cref{eq:div-charge,eq:div-current}), which, together with \cref{eq:potential} and an assumption of uniform and isotropic brain tissue, leads in both cases to a simple closed form equation to calculate the LFP $\phi$ (namely \cref{eq:phi-charge,eq:phi-current}).



\section{Model 1 (Gauss's law)}

The first model is a direct translation of the first of Maxwell's equations of electrogmagnetism (Gauss's law), at macroscopic scale:\footnote{Gauss's law in its pure form ($\div{\E} = \rho_\total / \eps_0$) considers both mobile and ``bound'' charges: $\rho_\total = \rho_\free + \rho_\bound$. In \cref{eq:div-charge}, we only explicitly consider mobile charges $\rho = \rho_\free$, while bound charges are subsumed in the tissue permittivity $\eps$. This simplification is allowed if we assume that ``tissue polarisation is proportional to electric field strength'' \cite{Feynman2013}. This is a common assumption in physics, and is largely valid for brain tissue in normal conditions \cite{Nunez2006}.}
%
\begin{equation}
\label{eq:div-charge}
\div{\eps \E} = \rho,
\end{equation}
%
where $\eps$ is the local tissue permittivity, in units of F/m and generally a 3 $\cross$ 3 matrix. When we assume the permittivity $\eps$ to be uniform and isotropic throughout the tissue\footnotemark{}, \cref{eq:div-charge} has the following solution for the LFP $\phi$ \cite{Feynman2013}:
%
\begin{equation}
\label{eq:phi-charge}
\phi = \frac{1}{4 \pi \eps} \int_V \frac{\rho \dd{V}}{r},
\end{equation}
%
where we summate over the entire brain volume $V$, with $\rho \dd{V}$ the surplus mobile charge in a small volume $\dd{V}$ of tissue, and $r$ the distance of this small volume to the point where $\phi$ is calculated.

\footnotetext{The uniform and isotropic permitivitty assumption is questionable. Roughly, brain tissue consists of a dense alternation of two types of tissue (see \cref{fig:neuropil}): the seawater-like fluid inside and in between cells, and the fatty membranes around cells, organelles, and vesicles. The former has a permittivity $\eps$ of about 15 times larger than the latter \cite{Marszalek1991,Weaver2003,Martinsen2015}. Additionally, the strongly organized anatomy of some brain regions may challenge the isotropy assumption. Nevertheless, this uniform and isotropic $\eps$ assumption is often made \cite{Nunez2006,Plonsey2007}.}

In other words, positive charge surplusses increase the nearby LFP, negative charge surplusses decrease it, and their effects summate linearly, weighted inverse-proportionally by distance. This model explains for example the resting ``membrane potential'' of neurons, where $\phi$ is lower inside the cell than outside: neurons at rest contain an excess of negative charges \cite{Dayan2001b}.

Although this model is arguably the most physically correct, it is often difficult to apply in practice: charge distributions $\rho$ are already quite complex in even the simplest of electronic circuits \cite[chapter 8]{Chabay2015}. For the highly entangled topologies of brain tissue (see \cref{fig:neuropil}), estimating charge distributions is intractable.

The next model is more useful in practice, as it calculates the LFP based not on charges, but on currents. Currents (and specifically transmembrane currents) are the bread and butter of electrophysiology \cite{Kandel2013a}.



\section{Model 2 (Current source density analysis)}
\label{sec:CSD}

This is the ``standard model of electric potentials in biological tissue'' \cite{Bedard2011}, known as current source density (CSD) analysis. Unlike the previous model, it is mostly empirical. In fact, the assumptions on which it rests are strictly incompatible with \cref{eq:div-charge,eq:phi-charge}, and lead to a paradox when considered jointly. \citefulls{Bedard2011} explore this paradox, and propose a more general formulation of CSD analysis that `solves' the paradox. In the following however, we will present the `classic' CSD model.


%
\begin{equation}
\label{eq:div-current}
\div{\sigma \E} = - \pdv{\rho}{t}
\end{equation}
%
%
\begin{equation}
\label{eq:CSD}
I := - \pdv{\rho}{t}
\end{equation}
%
%
\begin{equation}
\label{eq:phi-current}
\phi = \frac{1}{4 \pi \sigma} \int_V \frac{I \dd{V}}{r},
\end{equation}
%
Note the strong similarity with \cref{eq:phi-charge}.





% % \begin{align*}
% % \div{\J} &= - \pdv{\rho}{t} = I  \\
% % \J &= \sigma \E \\
% % \div{\E} &= \frac{-1}{\sigma} \pdv{\rho}{t} = \frac{\rho}{\eps} \\
% % \pdv{\rho}{t} &= - \frac{\sigma}{\eps} \rho \\
% % \\
% % \div{\grad{\phi}} &= \frac{1}{\sigma} \pdv{\rho}{t} \\
% % \laplacian{\phi} &= \frac{I}{\sigma} \\
% % \end{align*}

% % Destexhe paradox \& conclusie
