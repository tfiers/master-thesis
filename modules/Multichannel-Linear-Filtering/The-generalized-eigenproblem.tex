\subsection{The generalized eigenproblem}
\label{sec:generalized-eigenproblem}

An arbitrarily scaled vector $\w_i$ is a so called \emph{generalized eigenvector} (GEVec) for the ordered matrix pair $(\Rss, \Rnn)$ when the following holds:
%
\begin{equation}
\label{eq:generalized-eigenproblem}
\Rss \w_i = \lambda_i \Rnn \w_i,
\end{equation}
%
for some scalar $\lambda_i$, which is called the \emph{generalized eigenvalue} (GEVal) corresponding to $\w_i$. The largest scalar $\lambda_1$ for which \cref{eq:generalized-eigenproblem} holds is the `first' GEVal, and as mentioned before, the corresponding GEVec $\w_1$ is the solution $\what$ to \cref{eq:argmax_R}.

Since the 1960's, numerically stable algorithms exist that solve the generalised eigenproblem \cref{eq:generalized-eigenproblem} \cite{Golub2013}. A specialized algorithm is applicable when the input matrices are symmetric -- as is the case for $\Rss$ and $\Rnn$. This algorithm (based on a Cholesky factorization and the classical QR-algorithm for ordinary eigenproblems) is implemented in the LAPACK software package (as \texttt{ssygv} and \texttt{dsygv}), and can be easily applied using e.g. the \texttt{eig} function from MATLAB, or the \texttt{eigh} function from SciPy's \texttt{linalg} module.
