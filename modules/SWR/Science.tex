\section{Scientific importance}
\label{sec:science}

Sharp wave-ripples play an important role in learning and memory.\footnotemark{} The evidence for this has been accumulating over the years. This evidence has mostly been associative, although the kind of closed-loop experiments discussed in this thesis have recently been providing more direct evidence.

\footnotetext{In this section, when we refer to ``SWR's'', we mean ``SWR's, or the mechanism that generates them''. -- The SWR pattern in the LFP may not be more than an epiphenomenon. Additionally, when we refer to ``memory'', we more specifically mean so called \emph{declarative} memory. This excludes conditioned and ``instinctive'' emotional responses, ``muscle memory'' (habits and motor skills), and habituation or sensitization of the senses.}




\subsection{Overview of evidence}

We first summarize what is known about the importance of sharp wave-ripples.
% Most of these claims are supported by both clinical studies and animal experiments.
In the following subsections, we expand on each of these points.

\begin{enumerate}
\item The hippocampus -- where SWR's are recorded from -- is necessary for consolidating memories.
\item Memory consolidation happens mostly during sleep. SWR's also occur mostly during sleep.
\item The ideal firing frequency for strengthening synapses is the same as the ripple frequency.
\item Awake neural firing patterns are later replayed in the hippocampus. This replay occurs mostly during SWR's.
\item Forcibly silencing the hippocampus during ripples deteriorates performance on a newly learned task. Silencing the hippocampus outside ripples has no such effect.
\end{enumerate}




\subsection{Memory consolidation}

SWR's are thought to be pertinent mainly in the process of \emph{memory consolidation}, where certain short-term memories are stabilized into long-term memories. Single-cell recordings in primates have hinted that short-term memories exist as positive feedback loops of active neurons in the neocortex (``reverberations'', or persistent activity). Long-term memories on the other hand are likely to exist as strenghtened or newly built physical connections between neurons. \cite{Kandel2013,Bear2015}

The case of Henry Molaison (known as patient H.M. up until his death in 2008) illustrates the importance of memory consolidation, and of the role the hippocampus in this process. In an attempt to cure his epilepsy, most of Molaison's hippocampi and the adjacent entorhinal cortices\footnote{Main input to the hippocampus, relaying deeply processed sensory information from the rest of the neocortex.} were surgically removed (leaving the rest of the neocortex intact). Before the surgery, Molaison had no memory problems. After the surgery however -- although his epileptic seizures decreased -- he could not form new long-term memories.\footnotemark{} Curiously though, Molaison still had both intact short-term memory (remembering new information for up to minutes after), and largely intact long-term memory for events that happened before the surgery. These symptoms are consistent with a model of memory consolidation where both short-term and long-term memories are stored and retrieved by the neocortex, but where the hippocampus is needed to convert short-term into new long-term memories. \cite{Kandel2013}

\footnotetext{For example, the doctors working with Molaison had to re-introduce themselves on every occasion they saw him \cite{Bear2015}.}



% systems vs synaptic consolidation
% sleep, two stage-model



\subsection{Replay}

Quite recently, 




\subsection{Ripple disruption}

% \cite{Ego-Stengel2009,Girardeau2009,Jadhav2012,Girardeau2014,Talakoub2016}
