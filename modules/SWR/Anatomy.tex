\section{Anatomy of the SWR}
\label{sec:anatomy}

Sharp wave-ripples are a spatiotemporal pattern of the LFP of the \emph{hippocampus}. \Cref{fig:brain-anatomy} (left) shows the location of the hippocampus in the human and the rat brain.\footnote{Strictly, a brain has two hippocampi, one in each hemisphere, and mirrored with respect to each other. We will quite inconsistently refer to them in both singular and plural.}\footnotemark{} (The data analyzed in this thesis was recorded from a rat (see \cref{sec:recording}) which is why we focus on this animal).

\footnotetext{In the rat brain, the hippocampi are the size of two cooked grains of rice, and are shaped as two upright bananas joined at their ends. In the human brain, the hippocampi are about 7 cm long, lying lengthwise at the base of the brain, at approximately eye height \cite{Rao2012,Mai2015}. They are shaped like two seahorses (which is the namesake of the hippocampus -- Greek for seahorse), or like a ram's horns (which is the namesake of the CA regions in the hippocampus -- these stand for Cornu Ammonis, or Ammon's Horn, where Ammon is an Egyptian deity sometimes taking the form of a ram.)}

\begin{figure}
\img[0.6]{human-nissl}
\img[0.6]{human-HC-labelled}\\[4em]
\img[0.6]{rat-nissl}
\img[0.6]{rat-HC-labelled}\\[1em]
\captionn{Location of relevant brain structures in the human and the rat brain}{Coronal section of a human brain (top) and sagittal section of a rat brain (bottom). Both slices are Nissl-stained, which means that each purple dot represents a cell body. (This can be more clearly seen in \cref{fig:human-cells-comp}, which is a further zoom-in of the top-right panel). Lighter areas indicate regions that consist mostly of axons and dendrites. Black boxes mark location of the zoomed-in micrographs at the right hand side. Note the correspondence in shape and constituents between the human and the rat hippocampus. Note also the grossly enlarged neocortex in the human brain relative to the size of other brain structures, when compared with the rat brain. DG: dentate gyrus. CA1/2/3: Cornu Amonis region 1/2/3. Or: stratum oriens. Pyr: pyramidal cell layer. Rad: stratum radiatum. LM: stratum lacunosum-moleculare. Mol: molecular layer of the dentate gyrus. Human slice from \cite{Mai2008} (30 mm posterior to the anterior commisure), with \cite{Mai2015} as a guide for labelling. Rat slice from \cite{Paxinos2007} (plate 175 -- 3.4 mm lateral).}
\label{fig:brain-anatomy}
\end{figure}

More specifically, SWR's are observed in \emph{area CA1} of the hippocampus (see the right-hand side of \cref{fig:brain-anatomy}). This brain area is highly organized: most neuron cell bodies are concentrated in a thin layer, the \emph{pyramidal cell layer}. (This is the darkly colored C-shape in the stained brain slices). The dendrites of these neurons are located in the layers above and below this pyramidal cell layer, more specifically in the \emph{stratum oriens} above, and the \emph{stratum radiatum} and \emph{stratum lacunosum-moleculare} below. (These layers have a lighter color in the stained brain slices).

The ``ripple'' part of a sharp wave-ripple is observed in and around the pyramidal cell layer of CA1, and consists of a $\pm$ 40 ms long oscillation of the extracellular electric field potential, at a frequency of 100 -- 200 Hz (period of 5 -- 10 ms). See the time series in \cref{fig:recording-location} for two examples. See also \cref{tab:bands} for a comparison of reported ripple bands in the literature.

The ``sharp wave'' part is observed simultaneously, in all layers of CA1, but most markedly in the stratum radiatum. In the stratum radiatum and the stratum lacunosum-moleculare, the sharp wave consists of a strong potential drop, followed by a return to the baseline potential. Together these form a downwards peak of $\pm$ 40 ms wide. Throughout the pyramidal cell layer, the polarity of this peak gradually reverses. The sharp wave has near zero power in the middle of the pyramidal cell layer, and consists of an upwards peak near the top of the pyramidal cell layer and in the stratum oriens above.
