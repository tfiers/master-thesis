\section{Recording SWR's}
\label{sec:recording}

Sharp wave-ripples can be recorded by making a small opening in the skull and inserting an electrode in the brain, ending in area CA1. All analyses in this thesis were carried out on such a recording, made in 2014 by \citeauthor*{Michon2016}, and described in \cite{Michon2016}.

The data consists of a 34-minute long, multi-channel time-series of the extracellular electric field potential in area CA1 of a rat, made with a flexible, silicon-based, multi-electrode probe. (See \cref{fig:implant-photos} for photos of the full probe and the implant). Specifically, we used data from the probe labelled ``L2'' in \cite{Michon2016}. \Cref{fig:recording-location} shows the layout of the electrodes on this probe, and the approximate location of the probe with respect to the different structures in the hippocampus. The probe has 16 electrodes, 8 of which are placed in a linear array (spaced 50 \um{} apart), and 8 placed in a cluster (of 100 \um{} $\cross$ 40 \um{}, with an inter-electrode distance of $\pm$ 25 \um{}). The electrodes cover a total depth of 500 \um{} and span all layers of area CA1.
The recording was made while the rat was at rest. This makes it likely that  sharp wave-ripples are observed \cite{Buzsaki2015}. 
% Reference electrode where?

% todo: tets

We detected SWR's in this recording according to the offline procedure described in \cref{ch:offline}. \Cref{fig:SWR-stats} shows where the resulting SWR's occur along the recording, as well as the distribution of SWR durations. The last 40\% of the recording (i.e. the last 14 minutes) were used to evaluate online detection algorithms (as in \cref{ch:eval}). The other 60\% were used to train data-driven algorithms (\cref{ch:GEVec,ch:RNN}).\footnotemark{}

\footnotetext{The 60-40 division was chosen arbitrarily -- under the constraint that there are sufficient amounts of both training and test data.}

\begin{figure}
\img{SWR-locations} \\[0.8em]
\img{SWR-durations} \img{inter-SWR-intervals}
\captionn{Statistics of the recorded SWR's}{SWR locations and extents calculated offline according to the procedure described in \cref{ch:offline}.
\Top: SWR frequency along the duration of the recording. Kernel density estimate with Gaussian kernel, $\sigma = 4$ seconds.
\Bottom: Distribution of SWR duration, and duration between SWR's. (Boxplot whiskers indicate $1.5$ times the interquartile range).}
\label{fig:SWR-stats}
\end{figure}
