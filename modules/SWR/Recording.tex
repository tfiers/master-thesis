\begin{figure}
\img[1.3]{recording-location}\\[0.5em]
\captionn{Recording location}{Composite image showing estimated location of the probe tip in area CA1 of the rat hippocampus. Hippocampal regions (black text) and layers of region CA1 (grey text) are indicated to the left. Overlaid on the micrograph of a slice of rat hippocampus are a cartoon of a possible CA1 pyramidal neuron (purple), and an axon (orange) leaving a neuron in CA3 to synapse onto the apical dendrites of the CA1 neuron. Also overlaid is a photograph of the ``L-style'' probe from \cite{Michon2016}. Gold squares are its 16 electrodes. Time series (right) show example voltage recordings from five of these electrodes (which are marked with a white dot). The displayed data shows two sharp wave-ripple events. Note the ripples in and around the pyramidal cell layer (``stratum pyramidale''). Note also the phase reversal of the sharp wave: from strongly negative in the stratum radiatum, to near zero power in the middle of the pyramidal cell layer, to positive in the stratum oriens. (Coronal slice of rat hippocampus from \cite{Kjonigsen2011}. Location: 2.98 mm posterior to bregma. Slice stained for parvalbumin, which highlights interneuron somata and neurites. Anterior-posterior and medial-lateral location of the probe were estimated based on the location of probe ``L2'' given in \cite{Michon2016}. Depth of probe estimated from the characteristics of the time-series recorded from each electrode. CA1 pyramidal cell morphology from \cite{Amaral2007}).}
\label{fig:recording-location}
\end{figure}


\section{Recording sharp wave-ripples}
\label{sec:recording}

These sharp wave-ripples can be recorded by making a small opening in the skull and inserting an electrode in the brain, ending in area CA1. All analyses in this thesis were carried out on such a recording, made in 2014 by \citeauthor*{Michon2016}, and described in \citefull{Michon2016}.

The data consists of a 34-minute long, multi-channel time-series of the extracellular electric field potential in area CA1 of a rat, made with a flexible, silicon-based, multi-electrode probe. (See \cref{fig:implant-photos} for photos of the full probe and the implant). Specifically, we used data from the probe labelled ``L2'' in \cite{Michon2016}. \Cref{fig:recording-location} shows the layout of the electrodes on this probe, and the approximate location of the probe with respect to the different structures in the hippocampus. The probe has 16 electrodes, 8 of which are placed in a linear array (spaced 50 \um{} apart), and 8 placed in a cluster (of 100 \um{} $\cross$ 40 \um{}, with an inter-electrode distance of $\pm$ 25 \um{}). The electrodes cover a total depth of 500 \um{} and span all layers of area CA1.
The recording was made while the rat was at rest. This makes it likely that  sharp wave-ripples are observed \cite{Buzsaki2015}. 
% Reference electrode where?

% tets
