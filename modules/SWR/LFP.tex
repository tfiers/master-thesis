\section{The local field potential}
\label{sec:LFP}

Biological tissue contains a high number of mobile charges (mostly ions, with \ce{Na+}, \ce{Cl-}, and \ce{K+} the most abundant ones \cite{Martinsen2015a}). All these charges are electric monopoles, emitting electric fields that summate linearly. The potential of the resulting total electric field, measured extracellularly, is called the \emph{local field potential} (LFP) by neuroscientists -- especially when only frequencies below about 500 Hz are considered. The following paragraphs briefly discuss the relation between the original mobile charges and the LFP.

Given the locations of mobile charges, the electric field (and thus the LFP) is entirely described by the following two equations:\footnotemark{}
%
\begin{align}
\div{\eps_r \E}  &= \frac{\rho_\free}{\eps_0}   \label{eq:gauss} \\
\curl{\E}        &= \vb{0},                     \label{eq:irrot}
\end{align}
%
with $\E$ the total electric field, $\rho_\free$ the mobile charge density, and $\eps_r$ the relative permittivity (a tissue property also known as the dielectric constant; sometimes denoted by $\kappa$). The vector $\E$, the scalar $\rho_\free$, and the 3 $\cross$ 3 matrix $\eps_r$ are all time-dependent fields, i.e. defined for each time and location in the tissue. $\eps_0 \approx \SI{10}{\pF\per\m}$ is the distributed capacitance of free space. Finally, the LFP is a scalar field $\phi$ such that
%
\begin{equation}
\label{eq:potential}
\E = -\grad{\phi},
\end{equation}
%
i.e. such that the electric field points from locations of high potential to locations of lower potential.

\footnotetext{These equations follow from Maxwell's equations, using the following two simplifications \cite{Feynman2013}. First, we assume that the electric and magnetic fields do not vary too quickly over time, so that both fields become decoupled. This so called \emph{quasi-static} assumption is met in electrophysiological conditions \cite{Nunez2006,Plonsey2007}. Second, we assume that tissue polarisation is proportional to electric field strength -- a common assumption in physics, that is largely valid for brain tissue in normal conditions \cite{Nunez2006}. As a result, we only need to know the charge density $\rho_\free$ of mobile charges (mostly ions), and not of bound charges in the tissue.}

Although \cref{eq:gauss,eq:irrot} are quite correct theoretically, they are generally not useful to calculate or to think about the LFP. First, they require knowledge of the mobile charge distribution, which is highly complex.\footnotemark{} Second, because $\eps_r$ is location dependent, \cref{eq:gauss} has no closed form analytical solution, and needs instead to be simulated numerically.

\footnotetext{Charge distributions are already quite complex in even the simplest of electrical circuits \cite[chapter 8]{Chabay2015}. For the highly entangled topologies of neuropil (see \cref{fig:neuropil}), obtaining charge distributions is intractable.}


Electrophysiologists have 

