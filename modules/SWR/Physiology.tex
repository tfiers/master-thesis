\section{Physiology of the sharp wave-ripple}
\label{sec:physiology}

Sharp wave-ripples are observed during deep sleep (so called \emph{slow-wave sleep}) and during certain awake states, such as drinking, eating, grooming, and staying still. They then occur at a rate of between one every ten seconds, to twice per second \cite{Girardeau2011}. In contrast, SWR's almost never occur during exploratory behaviours such as walking or running, rearing\footnote{standing up on the hind legs}, or sniffing. (During these behaviours, the hippocampal LFP is dominated by a $\pm$ 4 Hz oscillation -- the so called $\theta$-state). SWR's have so far been observed in all investigated mammalian species, including humans. \cite{Buzsaki2015}

SWR's are generated as follows. 
