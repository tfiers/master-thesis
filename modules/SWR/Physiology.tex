\section{Physiology}
\label{sec:physiology}

Sharp wave-ripples are observed during deep sleep (so called \emph{slow-wave sleep}) and during certain awake states, such as drinking, eating, grooming, and staying still. They then occur at a rate of between one every ten seconds, to twice per second \cite{Girardeau2011}. In contrast, SWR's almost never occur during exploratory behaviours such as walking, running, sniffing, or rearing\footnote{standing up on the hind legs}. (During these behaviours, the hippocampal LFP is dominated by a $\pm$ 4 Hz oscillation -- the so called $\theta$-state). SWR's have so far been observed in all investigated mammalian species, including humans \cite{Buzsaki2015}.

SWR's are generated as follows \cite{Girardeau2011,Buzsaki2015}. Large groups of excitatory neurons in area CA3 of the hippocampus simultaneously activate; i.e. send an action potential along their axons. These axons (called ``Schaffer collaterals'') end up in the stratum radiatum of area CA1. There, they stimulate the dendrites of CA1 neurons: at each synapse, a small current enters the CA1 neuron. In accordance with the LFP model of \cref{sec:forward-CSD}, these small, simultaneous, inward currents result in a drop in the LFP. This is the sharp wave. (See \cref{fig:recording-location} for SWR examples, and the corresponding anatomy).

As described in \cref{sec:transmembrane-currents}, the active, inward currents in the stratum radiatum give rise to passive, outward currents in other parts of the CA1 neurons. These outward currents occur mostly in the cell bodies, which are concentrated in the pyramidal cell layer. This is why the `polarity' of the sharp wave reverses when going from the stratum radiatum, through the pyramidal cell layer, up to the stratum oriens: the outward currents result in an increase in the LFP. Around the bottom-middle of the pyramidal cell layer, the effects of the inward and the outward currents cancel out, resulting in an absence of the sharp wave. Nearer the stratum oriens, the outward currents dominate, resulting in the upwards sharp wave.

The ripple oscillation is generated through delayed inhibition. The so called CA1 \emph{pyramidal neurons}, that are activated in the sharp wave, synapse onto inhibitory \emph{interneurons}, which are also present in CA1. The axons of these interneurons synapse back onto the pyramidal cell bodies. This recurrent connection pattern enables the ripple oscillation: sharp wave-activated pyramidal cells activate interneurons, which silence the pyramidal cells. This deactivates the interneurons, allowing the pyramidal cells to activate again. The intrinsic frequency of this system is the ripple frequency, 100 -- 200 Hz. The LFP is low when the pyramidal cells are active (spikes and excited synapses correspond to inward currents). It is high when the interneurons are active (inhibited synapses correspond to outward currents).
