\chapterp{2p}{Introduction}

\sectionp{1p}{Closed-loop neuroscience}
\label{sec:BCI}

This thesis discusses software for a brain-computer interface (BCI).\footnotemark{} Most BCI research is focused on restoring or partially replacing a damaged nervous system after a stroke or a traumatic injury \cite{Krucoff2016}. The BCI discussed in this thesis however is used for fundamental brain research into memory \& learning, using laboratory animals such as rats or macaques.

\footnotetext{A BCI is a device that reads out brain activity, which is used to control some action. The brain receives feedback about the performed action, either through normal sensory input, or through direct neural stimulation, thereby forming a closed loop.}

On a high level, the discussed BCI works as follows: an electrode is surgically implanted into the animal's brain, where it records the electric field potential. During an experiment, the BCI continuously tests whether it can detect a particular pattern in the recording. (This pattern, called the \emph{sharp wave-ripple}, is discussed in \cref{ch:SWR}). Each detection of the pattern triggers a feedback action. This is often a current injection that temporarily disables the brain area where the pattern is recorded from.

This experimental technique enables scientists to test their hypotheses about sharp wave-ripples, the affected brain area, and learning \& memory in general \cite{Ego-Stengel2009,Girardeau2009,Jadhav2012,Girardeau2014,Talakoub2016}. This is discussed further in \cref{sec:science}.


\sectionp{1p}{Problem statement}
\sectionp{1p}{Thesis overview}
