\chapterp{9p}{Introduction}


\sectionp{1p}{Closed-loop brain-computer interfaces}
\label{sec:BCI}

This thesis discusses software for a brain-computer interface (BCI).\footnotemark{} Most BCI research is focused on restoring or partially replacing a damaged nervous system after a stroke or a traumatic injury \cite{Krucoff2016}. The BCI discussed in this thesis however is used for fundamental neuroscientific research into memory \& learning, using model animals such as rats or macaques.

\footnotetext{A BCI is a device that reads out brain activity, which is used to control some action. The brain receives feedback about the performed action, either through normal sensory input, or through direct neural stimulation.}

On a high level, the BCI works as follows: an electrode is surgically implanted into the animal's brain, where it records the electric field potential. During an experiment, the BCI continuously tests whether it can detect a particular pattern in the recording. (This pattern, called the \emph{sharp wave-ripple}, is discussed in \cref{sec:SWR}). Each detection of the pattern then triggers some action. This is often a current injection that temporarily disables the brain area where the pattern was recorded from \cite{Ego-Stengel2009,Girardeau2009,Jadhav2012,Girardeau2014,Talakoub2016}.


\sectionp{6p}{Sharp wave-ripples}
\label{sec:SWR}

Every 


    \subsectionp{2p}{Description}
    \subsectionp{2p}{Scientific importance}
    \subsectionp{1p}{Biophysics}
    \subsectionp{1p}{Closed-loop technology}
\sectionp{1p}{Problem statement}
\sectionp{1p}{Thesis overview}
