\chapter{Scientific importance of SWR's}
\label{ch:science}

Sharp wave-ripples play an important role in learning and memory.\footnotemark{} The evidence for this has been accumulating over the years, and has mostly been associative / correlational. The kind of closed-loop experiments discussed in this thesis however have recently been providing more direct evidence.

\footnotetext{In this section, when we refer to ``SWR's'', we mean ``SWR's, or the mechanism that generates them''. -- The SWR pattern in the LFP could be regarded as an epiphenomenon. Additionally, when we refer to ``memory'', we more specifically mean so called \emph{declarative} memory. This excludes conditioned and instinctive emotional responses, ``muscle memory'' (habits and motor skills), and habituation or sensitization of the senses.}

Most of this chapter is based on the review of \citeauthor{Girardeau2011} \cite{Girardeau2011}, and to a lesser extent on \cite{Buzsaki1986,Buzsaki2015,Olafsdottir2018,Purves2017}.




\section{Overview of evidence}

We first summarize what is known about the importance of sharp wave-ripples. In the following sections, we expand on each of these points.
%
\begin{enumerate}
\item The hippocampus -- where SWR's are recorded from -- is necessary for consolidating memories.
\item Memory consolidation happens mostly during sleep. SWR's also occur mostly during sleep.
\item The high firing frequency of neurons during SWR's is ideal for strengthening synapses.
\item Awake neural firing patterns are later replayed, in both the hippocampus and the neocortex. These replays occur mostly during SWR's.
\item Forcibly silencing the hippocampus during SWR's deteriorates performance on a newly learned task. Silencing the hippocampus outside SWR's has no such effect.
\end{enumerate}
%
The last two points have only been convincingly demonstrated in rats, but are likely true for other mammals as well.




\section{Memory consolidation}

SWR's are thought to be pertinent mainly in the process of \emph{memory consolidation}, where some short-term memories are stabilized into long-term memories. Single-cell recordings in primates have hinted that short-term memories exist as positive feedback loops of firing neurons in the neocortex (``reverberations'', or persistent activity). Long-term memories on the other hand are likely to exist as strenghtened or newly built physical connections between neurons in the neocortex. \cite{Kandel2013,Bear2015,Purves2017}

The case of Henry Molaison (known as ``patient H.M.'' up until his death in 2008) illustrates the importance of memory consolidation, and of the role of the hippocampus in this process. In an attempt to cure his epilepsy, most of Molaison's hippocampi and the adjacent entorhinal cortices were surgically removed, leaving the rest of the neocortex intact. Before the surgery, Molaison had no memory problems. After the surgery however -- although his epileptic seizures decreased -- he could not form new long-term memories.\footnotemark{} Curiously though, Molaison still had both intact short-term memory (remembering new information for up to minutes after), and largely intact long-term memory for events that happened before the surgery. Similar symptoms have been observed in other clinical cases and in lesioned animal experiments. These symptoms are consistent with a model of memory consolidation where both short-term and long-term memories are stored in and retrieved by the neocortex, but where the hippocampus is needed to convert short-term into new long-term memories. \cite{Kandel2013,Purves2017}

\footnotetext{For example, the doctors working with Molaison had to re-introduce themselves on every occasion they saw him \cite{Bear2015}.}

This memory consolidation is thought to take place mostly during sleep. In this so called \emph{two-stage model} of memory consolidation, new information is first input to the hippocampus during the awake state. In the second stage, during subsequent sleep, this information is consolidated `offline' to the neocortex, by the hippocampus. As mentioned in \cref{sec:physiology}, SWR's occur mostly during sleep.





\section{Hippocampo-cortical connections}
\label{sec:HC-neocortex}

At a more detailed anatomic and physiological level, there is indeed quite some evidence for this model of memory consolidation. The main input to the hippocampus is the entorhinal cortex, which relays deeply processed sensory information from the rest of the neocortex. In turn, the hippocampus has many output projections to the neocortex (in large part to the prefrontal and the anterior cingulate cortices, where both short- and long-term memories are thought to mostly reside).

It has been discovered in rats that during SWR's, neocortical neurons indeed tend to fire shortly after hippocampal neurons \cite{Wierzynski2009}. Additionally, awake firing patterns in the neocortex have been found to be \emph{replayed} later, specifically during SWR's \cite{Peyrache2009}. (More about replay in \cref{sec:replay}. That section discusses replay in the hippocampus itself. Hippocampal replay is more established in the literature than neocortical replay).

Finally, during alert wakefulness, the hippocampus contains higher levels of neuromodulators that enhance the influence of external inputs relative to internal activity. During sleep on the other hand, the hippocampus contains lower concentrations of such neuromodulators, and there is simply less sensory input. This favors endogenous activity; the highly recurrent connectivity of the hippocampus is then thought to spontaneously generate SWR's.




\begin{figure}
\img[1.2]{replay}
\captionn{Replay}{\emph{Adapted from \citeauthor{Girardeau2011} \cite{Girardeau2011}.} See text for details.}
\label{fig:replay}
\end{figure}

\section{Replay}
\label{sec:replay}

Around the turn of the millenium, a memory-related discovery was made in the hippocampus at a quite detailed and mechanistic level, yielding new evidence for the importance of sharp wave-ripples. This phenomenon, called \emph{replay}, was discovered through simultaneous recordings of the spiking activity of multiple individual neurons in the hippocampus.\footnotemark{} See \cref{fig:replay} for a schematic example of replay.

\footnotetext{Such recordings are classically made with tetrodes -- four tightly wound electrodes. More recently, multi-electrode neural probes are also being used. Each electrode picks up the spikes from multiple nearby neurons; and each neuron's spikes are visible on multiple electrodes. Each neuron's spikes leave a distinct signature on the multi-electrode recordings. The process of finding these signatures and using them to group recorded spikes by their generating neuron is called \emph{spike sorting}.}

When a rat is placed in a maze, there is a good chance that a given hippocampal neuron is a so called \emph{place cell}, meaning that it only fires around a certain location in the maze.\footnotemark{}. When the rat walks around the maze, a set of place cells will therefore fire in sequence, reflecting the path that the rat traversed.

\footnotetext{The area where it fires is called the neuron's \emph{place field}. To be precise, the neuron can fire anywhere in the maze, but has an increased firing \emph{rate} in its place field. Finding the rat's position based on neuron firing rates is called \emph{neural population decoding}, and is typically performed using a Bayesian framework to invert the empirical place field encodings $p(\text{fire} \given \text{position})$\cite{Kloosterman2012}.}

When the rat later rests, these place cell firing sequences occur again, even though the rat stays in the same place. These sequence reactivations are called \emph{replay}. They are compressed in time versus the original sequences, and, importantly, occur mostly during sharp wave-ripples. This time-scale means that reactivated neurons (and their downstream projections) are likely to become more tightly connected, as is explained in \cref{sec:synaptic-plasticity}.

It has been hypothesized that not only spatial trajectories are replayed during SWR's, but in general any type of episodic memory -- especially in primates \cite{Girardeau2011}.




\section{Synaptic plasticity}
\label{sec:synaptic-plasticity}

Long-term memories are formed by changing the strength of synapses between neurons, which is known as \emph{synaptic plasticity}.\footnotemark{}

\footnotetext{This strengthening is measured as an increased postsynaptic current or potential-difference, for an identical presynaptic spike; i.e. the `weight' or the gain of the connection has persistently increased.}

Long-term potentiation (LTP) is any biomolecular process that strengthens  synapses based on recent presynaptic spiking activity. It is often induced after the presynaptic neuron has fired at a relatively high frequency (a so called \emph{tetanus} or burst of spikes).
% Conversely, long-term depression (LTD) weakens synapses after prolonged low-frequency presynaptic spiking.

Spike-timing dependent plasticity (STDP) is a phenomenon where synapses are strengthened or weakened based on both pre- and postsynaptic firing activity. When the postsynaptic neuron fires shortly after the presynaptic neuron, the synapse is strengthened. When it fired shortly before the presynaptic neuron however, the synapse is weakened. STDP is a translation of Hebb's `rule' that ``neurons that fire together, wire together'' (extended with Hebb's intuitions about temporal precedence) \cite{Caporale2008}.

As mentioned earlier, many neurons in area CA1 of the hippocampus 1) fire at high frequency during SWR's (100--200 Hz), and 2) have projections (i.e. axons) to the neocortex. This is ideal to induce LTP in downstream neocortical neurons. Additionally, the temporal firing sequences replayed at ripple-speed in the hippocampus (see \cref{sec:replay}) are ideal to connect neurons through STDP, both within the hippocampus and within the neocortex.




\section{Planning \& decision-making}
\label{sec:planning}

We briefly mention a possible complementary role for SWR's and replay, besides learning \& memory.

Analyzing place cell spiking during awake SWR's revealed sped-up mental trajectories through the maze, similar to sleep replay. These awake place cell sequences are generally more varied than sleep replay sequences: the replay can be in reverse (i.e. the place cells activate in reverse order than the navigated path), especially when the rat is consuming its reward. Or the place cell sequence can seemingly trace out a path not navigated before (this has been termed ``preplay''). Additionally, while traversing a maze, place cell sequences have been observed to trace out paths ahead of the rat -- especially when the animal stops at a fork in the maze.

Such observations are less well established than sleep replay, but -- together with corroborating evidence from cognitive neuroscience\footnotemark{} -- point at a planning, imagination, and decision making-role for awake hippocampal replay and sharp wave-ripples.

\footnotetext{Cognitive neuroscience is, roughly, human neuroscience. Think psychological experiments, MRI machines, and clinical studies.}




\section{Ripple disruption}
\label{sec:disruption}

The above evidence for the importance of SWR's is correlational / associational. Recently however, more direct evidence is being provided by studies that perform real-time SWR detection and disruption \cite{Ego-Stengel2009,Girardeau2009,Jadhav2012,Girardeau2014,Kovacs2016,Talakoub2016}.

Such disruption is typically accomplished by inserting an electrode in the ventral hippocampal commisure (VHC), which is a tight bundle of axons connecting hippocampal neurons from both hemispheres. Stimulating this axon bundle by sending a current pulse through the electrode activates many hippocampal neurons, including inhibitory interneurons. These interneurons silence the hippocampus, such that none or very few spikes occur for $\pm$ 50 ms after stimulation. This means that, when the stimulation is applied during an SWR-event, the ripple oscillation disappears after the stimulation.

In 2009, two very similar studies were published that for the first time applied real-time ripple detection and disruption \cite{Ego-Stengel2009,Girardeau2009}. In both studies, rats had to learn to navigate a simple maze for a chocolate reward. After each trial through the maze, the rats were allowed to sleep for one hour in a separate box. During this sleep, hippocampal ripples were detected in real time and disrupted through VHC stimulation. After sleep, the rats had to navigate the same maze. This was repeated for several days.

In both studies, ripple disruption deteriorated performance when compared with the controls:\footnotemark{} more wrong turns were taken, navigating the maze took longer, and less sucessful runs were completed. Learning the maze over the course of days progressed more slowly. These results show quite directly the importance of SWR's for learning \& memory.

\footnotetext{The two studies used different controls: in \citefull*{Ego-Stengel2009}, trials on the maze where ripples were disrupted during subsequent sleep were alternated with trials on a second maze, where ripples were not disrupted during subsequent sleep. In \citefull{Girardeau2009}, only one type of maze was used, but the rats were grouped in three types: rats without implanted electrodes, rats with ripple disruption, and rats where the VHC was stimulated, but only after $\pm$ 100 ms had passed after online detection of a ripple, meaning that the detected ripple is not disrupted. This controls for non-ripple-associated effects of VHC stimulation. The rats with delayed VHC stimulation performed equally well as the unimplaneted rats.}

Other studies employing real-time ripple detection followed. As mentioned in \cref{sec:physiology}, SWR's also occur in the awake state. In \citefull{Jadhav2012}, awake ripples were disrupted, but ripples during sleep were not. This still resulted in a performance deficit for learning the maze, showing the importance of awake SWR's ``for learning and memory-guided decision-making''. In \citefull{Girardeau2014}, ripple disruption was used to show that the hippocampus can generate more ripples when needed. \citefull{Novitskaya2016} used ripple detection to trigger stimulation not of the VHC, but of the \emph{locus coeruleus}, a primitive and global activity regulator of the brain. In \citefull{Kovacs2016}, the hippocampus was silenced not by electrical, but by optogenetic stimulation. Finally, \citefull{Talakoub2016} showed that real-time ripple detection and disruption is not only feasible in rats, but also in primates.
